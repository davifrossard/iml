%%%%%%%%%%%%%%%%%%%%%%%%%%%%%%%%%%%%%%%%%%%%%%%%%%%%%%%%%%%%
%%  This Beamer template was created by Cameron Bracken.
%%  Anyone can freely use or modify it for any purpose
%%  without attribution.
%%
%%  Last Modified: January 9, 2009
%%

\documentclass[xcolor=x11names,compress]{beamer}

%% General document %%%%%%%%%%%%%%%%%%%%%%%%%%%%%%%%%%
\usepackage{graphicx}
\usepackage{tikz}
\usepackage[canadian]{babel}
\usepackage[utf8]{inputenc}
\usepackage{amsmath,amssymb}
\usepackage{mathtools}
\usepackage[ruled,vlined,linesnumbered]{algorithm2e}
\usepackage{epstopdf}
\usepackage{hyperref}
\usepackage[export]{adjustbox}
\usepackage{animate}
\usepackage{minted}
\graphicspath{{img/}}
%%%%%%%%%%%%%%%%%%%%%%%%%%%%%%%%%%%%%%%%%%%%%%%%%%%%%%


%% Beamer Layout %%%%%%%%%%%%%%%%%%%%%%%%%%%%%%%%%%
\useoutertheme[subsection=false,shadow]{miniframes}
\useinnertheme{default}
\usefonttheme{serif}
\usepackage{palatino}
\urlstyle{same}
\setbeamerfont{title like}{shape=\scshape}
\setbeamerfont{frametitle}{shape=\scshape}

\setbeamercolor*{lower separation line head}{bg=DeepSkyBlue4} 
\setbeamercolor*{normal text}{fg=black,bg=white} 
\setbeamercolor*{alerted text}{fg=red} 
\setbeamercolor*{example text}{fg=black} 
\setbeamercolor*{structure}{fg=black} 
 
\setbeamercolor*{palette tertiary}{fg=black,bg=black!10} 
\setbeamercolor*{palette quaternary}{fg=black,bg=black!10} 

\renewcommand{\(}{\begin{columns}}
\renewcommand{\)}{\end{columns}}
\newcommand{\<}[1]{\begin{column}{#1}}
\renewcommand{\>}{\end{column}}
\addtobeamertemplate{navigation symbols}{}{%
	\usebeamerfont{footline}%
	\usebeamercolor[fg]{footline}%
	\hspace{1.5ex}%
	\insertframenumber/\inserttotalframenumber
}
\usemintedstyle{tango}
%%%%%%%%%%%%%%%%%%%%%%%%%%%%%%%%%%%%%%%%%%%%%%%%%%


\begin{document}


%%%%%%%%%%%%%%%%%%%%%%%%%%%%%%%%%%%%%%%%%%%%%%%%%%%%%%
%%%%%%%%%%%%%%%%%%%%%%%%%%%%%%%%%%%%%%%%%%%%%%%%%%%%%%
\section{\scshape Introduction}
\begin{frame}
\title{Introduction to Machine Learning}
\subtitle{TensorFlow}
\author{
	Davi Frossard\\
	{\it Federal University of Espirito Santo \\ University of Toronto}\\
}
\date{
    \vspace{-2em}\\
    \includegraphics[width=0.5\columnwidth]{tf_logo}\\[-1ex]
    {\tiny Credit: {\itshape \url{https://www.tensorflow.org}}\\[-2ex]
    \textcolor{gray}{TensorFlow, the TensorFlow logo and any related marks are trademarks of Google Inc.}}
    \\
	\today
}
\titlepage
\end{frame}

%%%%%%%%%%%%%%%%%%%%%%%%%%%%%%%%%%%%%%%%%%%%%%%%%%%%%%
%%%%%%%%%%%%%%%%%%%%%%%%%%%%%%%%%%%%%%%%%%%%%%%%%%%%%%
\begin{frame}{Summary}
\tableofcontents
\end{frame}

%%%%%%%%%%%%%%%%%%%%%%%%%%%%%%%%%%%%%%%%%%%%%%%%%%%%%%
%%%%%%%%%%%%%%%%%%%%%%%%%%%%%%%%%%%%%%%%%%%%%%%%%%%%%%
\section{\scshape TensorFlow}
\begin{frame}{TensorFlow}
	\begin{itemize}
		\item Initially developed by the Google Brain Team.
		\item Supports Python and C++.
		\item Open sourced in November 9\textsuperscript{th}, 2015.
		\item Used by Google in Speech Recognition, Gmail, Google Photos, Search.
		\item Recently adopted by the DeepMind team (AlphaGo).
		\item Currently the most popular machine learning framework.
	\end{itemize}
\end{frame}

%%%%%%%%%%%%%%%%%%%%%%%%%%%%%%%%%%%%%%%%%%%%%%%%%%%%%%
%%%%%%%%%%%%%%%%%%%%%%%%%%%%%%%%%%%%%%%%%%%%%%%%%%%%%%
\begin{frame}{TensorFlow}
	\begin{center}
		\includegraphics[width=0.9\columnwidth]{tf_usage}
	\end{center}
\end{frame}

%%%%%%%%%%%%%%%%%%%%%%%%%%%%%%%%%%%%%%%%%%%%%%%%%%%%%%
%%%%%%%%%%%%%%%%%%%%%%%%%%%%%%%%%%%%%%%%%%%%%%%%%%%%%%
\subsection{Features}
\begin{frame}{Features}
	\begin{itemize}
		\item Automatic Differentiation!
		\item Same code can be deployed to GPU or CPU.
		\item Parallelization over multiple machines.
		\item Easy deployment to Android.
		\item TensorBoard - Visualize your architectures.
	\end{itemize}
\end{frame}


%%%%%%%%%%%%%%%%%%%%%%%%%%%%%%%%%%%%%%%%%%%%%%%%%%%%%%
%%%%%%%%%%%%%%%%%%%%%%%%%%%%%%%%%%%%%%%%%%%%%%%%%%%%%%
\subsection{TensorBoard}
\begin{frame}{TensorBoard}
	\begin{center}
		\includegraphics[width=0.9\columnwidth]{tf_graph}
	\end{center}
\end{frame}

%%%%%%%%%%%%%%%%%%%%%%%%%%%%%%%%%%%%%%%%%%%%%%%%%%%%%%
%%%%%%%%%%%%%%%%%%%%%%%%%%%%%%%%%%%%%%%%%%%%%%%%%%%%%%
\subsection{Computation Graph}
\begin{frame}{Computation Graph}
	\begin{itemize}
		\item TensorFlow works with computation graphs.
		\item Nodes are operations, edges are data.
		\item If \textbf{a} and \textbf{b} are tensors, simply doing \textbf{a}+\textbf{b} will just create a node and not return any numerical result immediately.
		\item Increases performance.
		\item Might feel strange at first but rapidly becomes natural.
	\end{itemize}
\end{frame}

%%%%%%%%%%%%%%%%%%%%%%%%%%%%%%%%%%%%%%%%%%%%%%%%%%%%%%
%%%%%%%%%%%%%%%%%%%%%%%%%%%%%%%%%%%%%%%%%%%%%%%%%%%%%%
\subsection{Code Example}
\begin{frame}[fragile]{Code Example}
	\begin{minted}[mathescape,
	gobble=1,
	linenos,
	fontsize=\scriptsize,
	framesep=2mm,
	frame=lines,]{python}
	
	import tensorflow as tf
	
	a = tf.Variable(1)
	b = tf.Variable(3)
	
	c = a + b
	d = a * b
	
	e = c + d
	
	sess = tf.InteractiveSession()
	init = tf.initialize_all_variables()
	init.run()
	e.eval()
	\end{minted}
\end{frame}

%%%%%%%%%%%%%%%%%%%%%%%%%%%%%%%%%%%%%%%%%%%%%%%%%%%%%%
%%%%%%%%%%%%%%%%%%%%%%%%%%%%%%%%%%%%%%%%%%%%%%%%%%%%%%
\begin{frame}{Code Example}
	\begin{center}
		\includegraphics[width=\columnwidth]{graph}
	\end{center}
\end{frame}

%%%%%%%%%%%%%%%%%%%%%%%%%%%%%%%%%%%%%%%%%%%%%%%%%%%%%%
%%%%%%%%%%%%%%%%%%%%%%%%%%%%%%%%%%%%%%%%%%%%%%%%%%%%%%
\subsection{Code Example 2}
\begin{frame}[fragile]{Code Example \hspace{1ex}{\footnotesize Slightly Different}}
	\begin{minted}[mathescape,
	gobble=1,
	linenos,
	fontsize=\scriptsize,
	framesep=2mm,
	frame=lines,]{python}
	
	import tensorflow as tf
	
	a = tf.placeholder(tf.float32, [None])
	b = tf.placeholder(tf.float32, [None])
	
	c = a + b
	d = a * b
	
	e = c + d
	
	sess = tf.InteractiveSession()
	init = tf.initialize_all_variables()
	init.run()
	e.eval({a: [0,1,2], b: [2,2,2]})
	\end{minted}
\end{frame}


%%%%%%%%%%%%%%%%%%%%%%%%%%%%%%%%%%%%%%%%%%%%%%%%%%%%%%
%%%%%%%%%%%%%%%%%%%%%%%%%%%%%%%%%%%%%%%%%%%%%%%%%%%%%%
\section{\scshape Linear Regression}
\begin{frame}[fragile]{Linear Regression in TensorFlow}
	\begin{minted}[mathescape,
	gobble=1,
	linenos,
	fontsize=\scriptsize,
	framesep=2mm,
	frame=lines,]{python}
	
	import numpy as np
	import matplotlib.pyplot as plt
	import tensorflow as tf
	
	# Generate Mock Data
	data_x = np.linspace(1, 8, 100)[:, np.newaxis]
	data_y = np.polyval([1, -14, 59, -70], data_x) + 1.5 * np.sin(data_x) + np.random.randn(100, 1)
	# --------
	
	model_order = 5
	data_x = np.power(data_x, range(model_order))
	data_x /= np.max(data_x, axis=0) # Normalize inputs
	
	order = np.random.permutation(len(data_x))
	portion = 0.2*len(data_x)
	test_x = data_x[order[:portion]]
	test_y = data_y[order[:portion]]
	train_x = data_x[order[portion:]]
	train_y = data_y[order[portion:]]
	\end{minted}
\end{frame}

%%%%%%%%%%%%%%%%%%%%%%%%%%%%%%%%%%%%%%%%%%%%%%%%%%%%%%
%%%%%%%%%%%%%%%%%%%%%%%%%%%%%%%%%%%%%%%%%%%%%%%%%%%%%%
\begin{frame}{Linear Regression in TensorFlow}
	\begin{center}
		\includegraphics[width=0.9\columnwidth]{data}
	\end{center}
\end{frame}

%%%%%%%%%%%%%%%%%%%%%%%%%%%%%%%%%%%%%%%%%%%%%%%%%%%%%%
%%%%%%%%%%%%%%%%%%%%%%%%%%%%%%%%%%%%%%%%%%%%%%%%%%%%%%
\begin{frame}[fragile]{Linear Regression in TensorFlow}
	\begin{minted}[mathescape,
	gobble=1,
	linenos,
	fontsize=\scriptsize,
	framesep=2mm,
	frame=lines,
	firstnumber=20,
	tabsize=4]{python}
	
	# Input and output placeholders
	inputs = tf.placeholder(tf.float32, [None, model_order])
	outputs = tf.placeholder(tf.float32, [None, 1])
	
	# Model definition
	W = tf.Variable(tf.zeros([model_order, 1], dtype=tf.float32))
	y = tf.matmul(inputs, W)
	
	# Optimizer settings
	learning_rate = tf.Variable(0.5, trainable=False)
	cost_op = tf.reduce_mean(tf.pow(y-outputs, 2))
	sgd = tf.train.GradientDescentOptimizer(learning_rate)
	train_op = sgd.minimize(cost_op)
	\end{minted}
\end{frame}

%%%%%%%%%%%%%%%%%%%%%%%%%%%%%%%%%%%%%%%%%%%%%%%%%%%%%%
%%%%%%%%%%%%%%%%%%%%%%%%%%%%%%%%%%%%%%%%%%%%%%%%%%%%%%
\begin{frame}[fragile]{Linear Regression in TensorFlow}
	\begin{center}
		\includegraphics[width=0.9\columnwidth]{lr_graph}
	\end{center}
\end{frame}


%%%%%%%%%%%%%%%%%%%%%%%%%%%%%%%%%%%%%%%%%%%%%%%%%%%%%%
%%%%%%%%%%%%%%%%%%%%%%%%%%%%%%%%%%%%%%%%%%%%%%%%%%%%%%
\begin{frame}[fragile]{Linear Regression in TensorFlow}
	\begin{minted}[mathescape,
	gobble=1,
	linenos,
	fontsize=\tiny,
	framesep=2mm,
	frame=lines,
	firstnumber=33,
	tabsize=4]{python}
	
	epochs = 1
	last_cost = 0
	max_epochs = 50000
	
	sess = tf.Session() # Create TensorFlow session
	print "Beginning Training"
	with sess.as_default():
		init = tf.initialize_all_variables()
		sess.run(init)
		while True:
			# Execute Gradient Descent
			sess.run(train_op, feed_dict={inputs: train_x, outputs: train_y})
			# Keep track of our performance
			if epochs%100==0:
				cost = sess.run(cost_op, feed_dict={inputs: train_x, outputs: train_y})
				print "Epoch: %d - Error: %.4f" %(epochs, cost)
				# Stopping Condition
				if abs(last_cost - cost) < 1e-3 or epochs > max_epochs:
					print "Converged."
					break
				last_cost = cost
			
			epochs += 1
	
		print "w =", W.eval()
		print "Test Cost =", cost_op.eval({inputs: test_x, outputs: test_y})
	\end{minted}
\end{frame}

%%%%%%%%%%%%%%%%%%%%%%%%%%%%%%%%%%%%%%%%%%%%%%%%%%%%%%
%%%%%%%%%%%%%%%%%%%%%%%%%%%%%%%%%%%%%%%%%%%%%%%%%%%%%%
\begin{frame}[fragile]{Linear Regression in TensorFlow}
	\begin{minted}[mathescape,
	gobble=1,
	linenos,
	fontsize=\scriptsize,
	framesep=2mm,
	frame=lines,
	firstnumber=59,
	tabsize=4]{python}
	
	y_model = np.polyval(w[::-1], np.linspace(0,1,200))
	plt.plot(np.linspace(0,1,200), y_model, c='g', label='Model')
	plt.scatter(train_x[:,1], train_y, c='b', label='Train Set')
	plt.scatter(test_x[:,1], test_y, c='r', label='Test Set')
	plt.grid()
	plt.legend(loc='upper left')
	plt.xlabel('X')
	plt.ylabel('Y')
	plt.xlim(0,1)
	plt.show()
	\end{minted}
\end{frame}

%%%%%%%%%%%%%%%%%%%%%%%%%%%%%%%%%%%%%%%%%%%%%%%%%%%%%%
%%%%%%%%%%%%%%%%%%%%%%%%%%%%%%%%%%%%%%%%%%%%%%%%%%%%%%
\begin{frame}{Linear Regression in TensorFlow}
	\begin{center}
		\includegraphics[width=0.9\columnwidth]{fitted}
	\end{center}
\end{frame}

\end{document}